\documentclass[a4paper]{article}

\usepackage{../styles/new-style}
\usepackage{float}

\newcommand{\hmwkTitle}{Tasakaalustatud puud} % Assignment title
\newcommand{\hmwkClass}{Algoritmid ja andmestruktuurid} % Course/class

\begin{document}

\textbf{Kodutöö esitamise tähtaeg: 5. november, 23:59}

{\center
\subsection*{AVL-puu}
}

Eelmises kodutöös implementeerisime sõnastiku, kasutadeks selleks paisktabelit. Selles kodutöös implementeerime sõnastiku kasutades \textbf{tasakaalustatud kahendotsimispuid}.

Kahendotsimispuu on kahendpuu, mille kõik alampuud täidavad veel üht tingimust: 

Kui puu juurtipus $A$ on kirje $a$, siis kõik puus olevad kirjed, mis on kirjest $a$ väiksemad asuvad $A$ vasakus alampuus ning kõik kirjed, mis on kirjest $a$ suuremad asuvad $A$ paremas alampuus. 

\begin{figure}[H]
\centering
\small
\begin{tikzpicture}
\tikzstyle{level 1}=[sibling distance=60mm] 
\tikzstyle{level 2}=[sibling distance=30mm] 
\tikzstyle{level 3}=[sibling distance=20mm]
\tikzstyle{inverted}=[rectangle,fill=black,rounded corners]
\node [draw,rounded corners] (1a){$25$}
	child {node [draw, rounded corners] (2a) {$ 13 $} 
		child {node [draw, rounded corners] (3a) {$ 9 $} 
			child{ node [rectangle,draw, rounded corners] (4a) {$3$}
				child{ node [rectangle, draw, rounded corners] (5a) {$1$}}
				child{ node [rectangle, draw, rounded corners] (5b) {$7$}}
			}
			child{ node [rectangle,draw, rounded corners] (4b) {$11$}
			}
		}	
		child {node [rectangle,draw,rounded corners] (3b) {$23$} 
			child{edge from parent[white]} 
			child{ node [rectangle, draw, rounded corners] (4d) {$24$}}
		}
	}
	child {node [rectangle,draw,rounded corners] (2b) {$37$} 
		child {node [rectangle,draw,rounded corners] (3c) {$33$} 
			child{ node [rectangle, draw, rounded corners] (4e) {$31$}}
			child{ node [rectangle, draw, rounded corners] (4f) {$35$}}
		}
		child {node [rectangle,draw,rounded corners] (3d) {$52$}}		
	};
	


\end{tikzpicture}
\normalsize
\caption{Näide kahendotsimispuust, mille kirjeteks on täisarvud.}
\end{figure}


Ütleme, et kahendotsimispuu on \textbf{tasakaalus}, kui tema alampuude kõrguste vahe on ülimalt üks.

Kahendotsimispuud, kuhu saab elemente lisada ja kust saab elemente eemaldada nii, et puu jääb tasakaalu, nimetatake tasakaalustatud kahendotsimispuuks. 

Selles praktikumis keskendume ühele tasakaalustatud kahendotsimispuu implementatsioonile, AVL-puule. Praktikumijuhendajaga kokkuleppel on aga võimalik ka implementeerida mõni muu tasakaalus kahendotsimispuu. Tuntum alternatiiv on \textbf{puna-must puu}, \url{https://en.wikipedia.org/wiki/Red-black_tree}.



\begin{problem}
\textbf{Ülesanne}

Kasutades tasakaalustatud kahendpuud, implementeerida liides \textit{TreeMap}.
\end{problem}

Liidesed on kättesaadavad aaddressil \url{https://github.com/ut-aa/aa2016-lab5}.

\end{document}