\documentclass[a4paper]{article}

\usepackage{../styles/new-style}

\newcommand{\hmwkTitle}{Paisktabel} % Assignment title
\newcommand{\hmwkClass}{Algoritmid ja andmestruktuurid} % Course/class

\begin{document}

\textbf{Kodutöö esitamise tähtaeg: 22. oktoober, 23:59}


{\center
\subsection*{Paisktabel}
}


\begin{problem}
\textbf{Ülesanne 1} (40\%)

Implementeerida liides \textit{HashTable}.
\begin{enumerate}
\item 
Realiseerida lahtise addresseerimise ja lineaarse kompimisega paisktabel vastavalt liidesele.

\item Implementeerida paisktabel kimbumeetodil vastavalt liidesele.
\end{enumerate}
\end{problem}

Paiskfunktsiooniks võib võtta näiteks funktsiooni 

\[h(k) = [m\cdot (k \cdot T  - [k \cdot T])]\]

kus $k$ on võti, $m$ tabeli pikkus ja $T = \frac{\sqrt{5} - 1}{2}$, sulud $[...]$ tähistavad täisosa.

\begin{problem}
\textbf{Ülesanne 2} (20\%)

Viia läbi järgmised mõõtmised:

\begin{itemize}
\item fikseerida paisktabeli pikkus (nt 1000);

\item genereerida täisarvujärjend pikkusega $10\%, 20\%, \cdots , 90\%, 99\%$ paisktabeli pikkusest, elementideks juhuslikud täisarvud (nt 1-st 10000-ni);

\item paigutada järjendi elemendid paisktabelisse kummalgi meetodil;

\item genereerida teatud arv (nt 1000) juhuslikke täisarve samades piirides nagu
järjendi elemendid ja iga täisarvu puhul lugeda kokku, mitu võrdlemist tehakse
selle täisarvu otsimisel kummastki paisktabelist, eristades seejuures
edukat ja mitteedukat otsingut.
\end{itemize}

Tulemused esitada graafikutena, kus $x$-teljel on paisktabeli täituvus ($10\%,
20\%, \cdots , 90\%, 99\%)$ ja $y$-teljel keskmine võrdlemiste arv eduka otsingu korral ja mitteeduka otsingu korral kummastki paisktabelist.
\end{problem}



\begin{problem}
\textbf{Ülesanne 3} (30\%)

Implementeerida liides \textit{Sorter}.

Realiseerida järjendi sortimine vastavalt liidesele kimbumeetodiga. Paiskfunktsiooniks võtta ühtlane paiskamine.
\end{problem}


\begin{problem}
\textbf{Ülesanne 4} (10\%)

Võrrelda eelnevas ülesandes realiseeritud sortimismeetodi töökiirust mõne kiirema sortimismeetodi (keerukusega
$O(n\log n)$) kiirusega sorditava järjendi mitmesuguste pikkuste korral.
Järjendi elementideks valida nt $1000, \dots , 100000$ juhuslikku täisarvu 1 ja 100000 vahelt.

\medskip
Saadud tulemuste põhjal koostada graafik, kus x-teljel on järjendi pikkus ja
y-teljel kummagi sortimismeetodi tööaeg.
\end{problem}

Liidesed kättesaadavad aaddressil \url{https://github.com/ut-aa/aa2016-lab4}.

\end{document}